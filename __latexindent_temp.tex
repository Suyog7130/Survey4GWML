% Encoding: UTF-8

@article{foley2019gravity,
  title={Gravity and Light: Combining Gravitational Wave and Electromagnetic Observations in the 2020s},
  author={Foley, Ryan J and Alexander, KD and Andreoni, I and Arcavi, I and Auchettl, K and Barnes, J and Baym, G and Bellm, EC and Beloborodov, AM and Blagorodnova, N and others},
  journal={arXiv preprint arXiv:1903.04553},
  year={2019},
  url={https://arxiv.org/abs/1903.04553},
}

@article{eadie2019realizing,
  title={Realizing the potential of astrostatistics and astroinformatics},
  author={Eadie, Gwendolyn and Loredo, Thomas J and Mahabal, Ashish A and Siemiginowska, Aneta and Feigelson, Eric and Ford, Eric B and Djorgovski, SG and Graham, Matthew and Ivezic, Zeljko and Borne, Kirk and others},
  journal={arXiv preprint arXiv:1909.11714},
  year={2019},
  url={https://arxiv.org/abs/1903.11714},
}

@article{allen2019deep,
  title={Deep Learning for Multi-Messenger Astrophysics: A Gateway for Discovery in the Big Data Era},
  author={Allen, Gabrielle and Andreoni, Igor and Bachelet, Etienne and Berriman, G Bruce and Bianco, Federica B and Biswas, Rahul and Kind, Matias Carrasco and Chard, Kyle and Cho, Minsik and Cowperthwaite, Philip S and others},
  journal={arXiv preprint arXiv:1902.00522},
  year={2019},
  url={https://arxiv.org/abs/1902.00522},
}


@Article{huerta2019enabling,
  author   = {Huerta, E. A. and Allen, Gabrielle and Andreoni, Igor and Antelis, Javier M. and Bachelet, Etienne and Berriman, G. Bruce and Bianco, Federica B. and Biswas, Rahul and Carrasco Kind, Matias and Chard, Kyle and Cho, Minsik and Cowperthwaite, Philip S. and Etienne, Zachariah B. and Fishbach, Maya and Forster, Francisco and George, Daniel and Gibbs, Tom and Graham, Matthew and Gropp, William and Gruendl, Robert and Gupta, Anushri and Haas, Roland and Habib, Sarah and Jennings, Elise and Johnson, Margaret W. G. and Katsavounidis, Erik and Katz, Daniel S. and Khan, Asad and Kindratenko, Volodymyr and Kramer, William T. C. and Liu, Xin and Mahabal, Ashish and Marka, Zsuzsa and McHenry, Kenton and Miller, J. M. and Moreno, Claudia and Neubauer, M. S. and Oberlin, Steve and Olivas, Alexander R. and Petravick, Donald and Rebei, Adam and Rosofsky, Shawn and Ruiz, Milton and Saxton, Aaron and Schutz, Bernard F. and Schwing, Alex and Seidel, Ed and Shapiro, Stuart L. and Shen, Hongyu and Shen, Yue and Singer, Leo P. and Sipocz, Brigitta M. and Sun, Lunan and Towns, John and Tsokaros, Antonios and Wei, Wei and Wells, Jack and Williams, Timothy J. and Xiong, Jinjun and Zhao, Zhizhen},
  journal  = {Nature Reviews Physics},
  title    = {Enabling real-time multi-messenger astrophysics discoveries with deep learning},
  year     = {2019},
  issn     = {2522-5820},
  number   = {10},
  pages    = {600--608},
  volume   = {1},
  abstract = {Multi-messenger astrophysics is a fast-growing, interdisciplinary field that combines data, which vary in volume and speed of data processing, from many different instruments that probe the Universe using different cosmic messengers: electromagnetic waves, cosmic rays, gravitational waves and neutrinos. In this Expert Recommendation, we review the key challenges of real-time observations of gravitational wave sources and their electromagnetic and astroparticle counterparts, and make a number of recommendations to maximize their potential for scientific discovery. These recommendations refer to the design of scalable and computationally efficient machine learning algorithms; the cyber-infrastructure to numerically simulate astrophysical sources, and to process and interpret multi-messenger astrophysics data; the management of gravitational wave detections to trigger real-time alerts for electromagnetic and astroparticle follow-ups; a vision to harness future developments of machine learning and cyber-infrastructure resources to cope with the big-data requirements; and the need to build a community of experts to realize the goals of multi-messenger astrophysics.},
  refid    = {Huerta2019},
  url      = {https://doi.org/10.1038/s42254-019-0097-4},
}


@Article{huerta2019supporting,
  author   = {Huerta, E. A. and Haas, Roland and Jha, Shantenu and Neubauer, Mark and Katz, Daniel S.},
  journal  = {Computing and Software for Big Science},
  title    = {Supporting High-Performance and High-Throughput Computing for Experimental Science},
  year     = {2019},
  issn     = {2510-2044},
  number   = {1},
  pages    = {5},
  volume   = {3},
  abstract = {The advent of experimental science facilities--instruments and observatories, such as the Large Hadron Collider, the Laser Interferometer Gravitational Wave Observatory, and the upcoming Large Synoptic Survey Telescope --has brought about challenging, large-scale computational and data processing requirements. Traditionally, the computing infrastructure to support these facility’s requirements were organized into separate infrastructure that supported their high-throughput needs and those that supported their high-performance computing needs. We argue that to enable and accelerate scientific discovery at the scale and sophistication that is now needed, this separation between high-performance computing and high-throughput computing must be bridged and an integrated, unified infrastructure provided. In this paper, we discuss several case studies where such infrastructure has been implemented. These case studies span different science domains, software systems, and application requirements as well as levels of sustainability. A further aim of this paper is to provide a basis to determine the common characteristics and requirements of such infrastructure, as well as to begin a discussion of how best to support the computing requirements of existing and future experimental science facilities.},
  refid    = {Huerta2019},
  url      = {https://doi.org/10.1007/s41781-019-0022-7},
}

@article{fluke2020surveying,
  title={Surveying the reach and maturity of machine learning and artificial intelligence in astronomy},
  author={Fluke, Christopher J and Jacobs, Colin},
  journal={Wiley Interdisciplinary Reviews: Data Mining and Knowledge Discovery},
  volume={10},
  number={2},
  pages={e1349},
  year={2020},
  publisher={Wiley Online Library},
  url={https://arxiv.org/abs/1912.02934},
}

@article{huerta2020convergence,
  title={Convergence of Artificial Intelligence and High Performance Computing on NSF-supported Cyberinfrastructure},
  author={Huerta, EA and Khan, Asad and Davis, Edward and Bushell, Colleen and Gropp, William D and Katz, Daniel S and Kindratenko, Volodymyr and Koric, Seid and Kramer, William TC and McGinty, Brendan and others},
  journal={arXiv preprint arXiv:2003.08394},
  year={2020},
  url={https://arxiv.org/abs/2003.08394},
}

@article{cuoco2020enhancing,
  title={Enhancing Gravitational-Wave Science with Machine Learning},
  author={Cuoco, Elena and Powell, Jade and Cavagli{\`a}, Marco and Ackley, Kendall and Bejger, Michal and Chatterjee, Chayan and Coughlin, Michael and Coughlin, Scott and Easter, Paul and Essick, Reed and others},
  journal={arXiv preprint arXiv:2005.03745},
  year={2020},
  url={https://arxiv.org/abs/2005.03745},
}


@Article{zdeborova2020understanding,
  author   = {Zdeborová, Lenka},
  journal  = {Nature Physics},
  title    = {Understanding deep learning is also a job for physicists},
  year     = {2020},
  issn     = {1745-2481},
  number   = {6},
  pages    = {602--604},
  volume   = {16},
  abstract = {Automated learning from data by means of deep neural networks is finding use in an ever-increasing number of applications, yet key theoretical questions about how it works remain unanswered. A physics-based approach may help to bridge this gap.},
  refid    = {Zdeborová2020},
  url      = {https://doi.org/10.1038/s41567-020-0929-2},
}

@article{lightman2006prospects,
	doi = {10.1088/1742-6596/32/1/010},
	url = {https://doi.org/10.1088%2F1742-6596%2F32%2F1%2F010},
	year = 2006,
	month = {mar},
	publisher = {{IOP} Publishing},
	volume = {32},
	pages = {58--65},
	author = {M Lightman and J Thurakal and J Dwyer and R Grossman and P Kalmus and L Matone and J Rollins and S Zairis and S M{\'{a}}rka},
	title = {Prospects of gravitational wave data mining and exploration via evolutionary computing},
	journal = {Journal of Physics: Conference Series},
	abstract = {Techniques of evolutionary computing have proven useful for a diverse array of fields in science and engineering. Because of the expected low signal to noise ratio of LIGO data and incomplete knowledge of gravitational waveforms, evolutionary computing is an excellent candidate for LIGO data analysis studies. Using the evolutionary computing methods of genetic algorithms and genetic programming, we have developed, as a proof of principle, search algorithms that are effective at finding sine-gaussian signals hidden in noise while maintaining a small false alarm rate. Because we used realistic LIGO noise as a training ground, the algorithms we have evolved should be well suited to detecting signals in actual LIGO data, as well as in simulated noise. These algorithms have continuously improved during the five days of their evolution and are expected to improve further the more they are evolved. The top performing algorithms from generation 100 and 199 were benchmarked using gaussian white noise to illustrate their performance and the improvement over 100 generations.}
}

@article{essick2013optimizing,
	doi = {10.1088/0264-9381/30/15/155010},
	url = {https://doi.org/10.1088%2F0264-9381%2F30%2F15%2F155010},
	year = 2013,
	month = {jun},
	publisher = {{IOP} Publishing},
	volume = {30},
	number = {15},
	pages = {155010},
	author = {R Essick and L Blackburn and E Katsavounidis},
	title = {Optimizing vetoes for gravitational-wave transient searches},
	journal = {Classical and Quantum Gravity},
	abstract = {Interferometric gravitational-wave detectors like LIGO, GEO600 and Virgo record a surplus of information above and beyond possible gravitational-wave events. These auxiliary channels capture information about the state of the detector and its surroundings which can be used to infer potential terrestrial noise sources of some gravitational-wave-like events. We present an algorithm addressing the ordering (or equivalently optimizing) of such information from auxiliary systems in gravitational-wave detectors to establish veto conditions in searches for gravitational-wave transients. The procedure was used to identify vetoes for searches for unmodeled transients by the LIGO and Virgo collaborations during their science runs from 2005 through 2007. In this work we present the details of the algorithm; we also use a limited amount of data from LIGO's past runs in order to examine the method, compare it with other methods, and identify its potential to characterize the instruments themselves. We examine the dependence of receiver operating characteristic curves on the various parameters of the veto method and the implementation on real data. We find that the method robustly determines important auxiliary channels, ordering them by the apparent strength of their correlations to the gravitational-wave channel. This list can substantially reduce the background of noise events in the gravitational-wave data. In this way it can identify the source of glitches in the detector as well as assist in establishing confidence in the detection of gravitational-wave transients.}
}

@article{biswas2013application,
  title = {Application of machine learning algorithms to the study of noise artifacts in gravitational-wave data},
  author = {Biswas, Rahul and Blackburn, Lindy and Cao, Junwei and Essick, Reed and Hodge, Kari Alison and Katsavounidis, Erotokritos and Kim, Kyungmin and Kim, Young-Min and Le Bigot, Eric-Olivier and Lee, Chang-Hwan and Oh, John J. and Oh, Sang Hoon and Son, Edwin J. and Tao, Ye and Vaulin, Ruslan and Wang, Xiaoge},
  journal = {Phys. Rev. D},
  volume = {88},
  issue = {6},
  pages = {062003},
  numpages = {24},
  year = {2013},
  month = {Sep},
  publisher = {American Physical Society},
  doi = {10.1103/PhysRevD.88.062003},
  url = {https://link.aps.org/doi/10.1103/PhysRevD.88.062003}
}


@article{baker2015multivariate,
  title = {Multivariate classification with random forests for gravitational wave searches of black hole binary coalescence},
  author = {Baker, Paul T. and Caudill, Sarah and Hodge, Kari A. and Talukder, Dipongkar and Capano, Collin and Cornish, Neil J.},
  journal = {Phys. Rev. D},
  volume = {91},
  issue = {6},
  pages = {062004},
  numpages = {26},
  year = {2015},
  month = {Mar},
  publisher = {American Physical Society},
  doi = {10.1103/PhysRevD.91.062004},
  url = {https://link.aps.org/doi/10.1103/PhysRevD.91.062004}
}


@TechReport{2013RuslanVauliniDQRealTime,
  author = {Ruslan Vaulin, Lindy Blackburn, Reed Essick and Katsavounidis, Erik},
  title  = {iDQ: The Real-Time Pipeline for Glitch Identification},
  year   = {2013},
  institution = {MIT},
  publisher={LIGO Document G1300253-v1 (This document is not publicly accessible.)},
  url={https://dcc.ligo.org/LIGO-G1300253},
}


@article{davis2020utilizing,
	doi = {10.1088/1361-6382/ab91e6},
	url = {https://doi.org/10.1088%2F1361-6382%2Fab91e6},
	year = 2020,
	month = {jun},
	publisher = {{IOP} Publishing},
	volume = {37},
	number = {14},
	pages = {145001},
	author = {Derek Davis and Laurel V White and Peter R Saulson},
	title = {Utilizing {aLIGO} glitch classifications to validate gravitational-wave candidates},
	journal = {Classical and Quantum Gravity},
	abstract = {Advanced LIGO data contains numerous noise transients, or ‘glitches’, that have been shown to reduce the sensitivity of matched filter searches for gravitational waves from compact binaries. These glitches increase the rate at which random coincidences occur, which reduces the significance of identified gravitational-wave events. The presence of these transients has precipitated extensive work to establish that observed gravitational wave events are astrophysical in nature. We discuss the response of the PyCBC search for gravitational waves from stellar mass binaries to various common glitches that were observed during advanced LIGO’s first and second observing runs. We show how these transients can mimic waveforms from compact binary coalescences and quantify the likelihood that a given class of glitches will create a trigger in the search pipeline. We explore the specific waveform parameters that are most similar to different glitch classes and demonstrate how knowledge of these similarities can be used when evaluating the significance of gravitational-wave candidates.}
}


@article{powell2015classification,
	doi = {10.1088/0264-9381/32/21/215012},
	url = {https://doi.org/10.1088%2F0264-9381%2F32%2F21%2F215012},
	year = 2015,
	month = {oct},
	publisher = {{IOP} Publishing},
	volume = {32},
	number = {21},
	pages = {215012},
	author = {Jade Powell and Daniele Trifir{\`{o}} and Elena Cuoco and Ik Siong Heng and Marco Cavagli{\`{a}}},
	title = {Classification methods for noise transients in advanced gravitational-wave detectors},
	journal = {Classical and Quantum Gravity},
	abstract = {Noise of non-astrophysical origin will contaminate science data taken by the advanced laser interferometer gravitational-wave observatory and advanced Virgo gravitational-wave detectors. Prompt characterization of instrumental and environmental noise transients will be critical for improving the sensitivity of the advanced detectors in the upcoming science runs. During the science runs of the initial gravitational-wave detectors, noise transients were manually classified by visually examining the time–frequency scan of each event. Here, we present three new algorithms designed for the automatic classification of noise transients in advanced detectors. Two of these algorithms are based on principal component analysis. They are principal component analysis for transients and an adaptation of LALInference burst. The third algorithm is a combination of an event generator called wavelet detection filter and machine learning techniques for classification. We test these algorithms on simulated data sets, and we show their ability to automatically classify transients by frequency, signal to noise ratio and waveform morphology.}
}

@article{powell2017classification,
	doi = {10.1088/1361-6382/34/3/034002},
	url = {https://doi.org/10.1088%2F1361-6382%2F34%2F3%2F034002},
	year = 2017,
	month = {jan},
	publisher = {{IOP} Publishing},
	volume = {34},
	number = {3},
	pages = {034002},
	author = {Jade Powell and Alejandro Torres-Forn{\'{e}} and Ryan Lynch and Daniele Trifir{\`{o}} and Elena Cuoco and Marco Cavagli{\`{a}} and Ik Siong Heng and Jos{\'{e}} A Font},
	title = {Classification methods for noise transients in advanced gravitational-wave detectors {II}: performance tests on Advanced {LIGO} data},
	journal = {Classical and Quantum Gravity},
	abstract = {The data taken by the advanced LIGO and Virgo gravitational-wave detectors contains short duration noise transients that limit the significance of astrophysical detections and reduce the duty cycle of the instruments. As the advanced detectors are reaching sensitivity levels that allow for multiple detections of astrophysical gravitational-wave sources it is crucial to achieve a fast and accurate characterization of non-astrophysical transient noise shortly after it occurs in the detectors. Previously we presented three methods for the classification of transient noise sources. They are Principal Component Analysis for Transients (PCAT), Principal Component LALInference Burst (PC-LIB) and Wavelet Detection Filter with Machine Learning (WDF-ML). In this study we carry out the first performance tests of these algorithms on gravitational-wave data from the Advanced LIGO detectors. We use the data taken between the 3rd of June 2015 and the 14th of June 2015 during the 7th engineering run (ER7), and outline the improvements made to increase the performance and lower the latency of the algorithms on real data. This work provides an important test for understanding the performance of these methods on real, non stationary data in preparation for the second advanced gravitational-wave detector observation run, planned for later this year. We show that all methods can classify transients in non stationary data with a high level of accuracy and show the benefits of using multiple classifiers.}
}


@phdthesis{powell2017model,
  title={Model selection for gravitational-wave transient sources},
  author={Powell, Jade},
  year={2017},
  school={University of Glasgow},
  url={http://theses.gla.ac.uk/8259/},
}


@article{cuoco2018strategy, 
  place={Country unknown/Code not available}, 
  title={Strategy for signal classification to improve data quality for Advanced Detectors gravitational-wave searches}, 
  volume={Nuovo Cim. C40}, 
  url={http://par.nsf.gov/biblio/10061620}, 
  DOI={10.1393/ncc/i2017-17124-4}, 
  abstractNote={Noise of non-astrophysical origin contaminates science data taken by the Advanced Laser Interferometer Gravitational-wave Observatory and Advanced Virgo gravitational-wave detectors. Characterization of instrumental and environmental noise transients has proven critical in identifying false positives in the first aLIGO observing run O1. In this talk, we present three algorithms designed for the automatic classification of non-astrophysical transients in advanced detectors. Principal Component Analysis for Transients (PCAT) and an adaptation of LALInference Burst (PC-LIB) are based on Principal Component Analysis. The third algorithm is a combination of a glitch finder called Wavelet Detection Filter (WDF) and unsupervised machine learning techniques for classification.}, 
  number={3}, 
  journal={Proceedings, 11th Workshop on Science with the New generation of High Energy Gamma-ray Experiments (SciNeGHE 2016) : Pisa, Italy, October 18-21, 2016}, 
  author={Elena Cuoco}, 
  year={2018}, 
  month={Jan},
}

@article{razzano2018image,
	doi = {10.1088/1361-6382/aab793},
	url = {https://doi.org/10.1088%2F1361-6382%2Faab793},
	year = 2018,
	month = {apr},
	publisher = {{IOP} Publishing},
	volume = {35},
	number = {9},
	pages = {095016},
	author = {Massimiliano Razzano and Elena Cuoco},
	title = {Image-based deep learning for classification of noise transients in gravitational wave detectors},
	journal = {Classical and Quantum Gravity},
	abstract = {The detection of gravitational waves has inaugurated the era of gravitational astronomy and opened new avenues for the multimessenger study of cosmic sources. Thanks to their sensitivity, the Advanced LIGO and Advanced Virgo interferometers will probe a much larger volume of space and expand the capability of discovering new gravitational wave emitters. The characterization of these detectors is a primary task in order to recognize the main sources of noise and optimize the sensitivity of interferometers. Glitches are transient noise events that can impact the data quality of the interferometers and their classification is an important task for detector characterization. Deep learning techniques are a promising tool for the recognition and classification of glitches. We present a classification pipeline that exploits convolutional neural networks to classify glitches starting from their time-frequency evolution represented as images. We evaluated the classification accuracy on simulated glitches, showing that the proposed algorithm can automatically classify glitches on very fast timescales and with high accuracy, thus providing a promising tool for online detector characterization.}
}


@article{george2018classification,
  title = {Classification and unsupervised clustering of LIGO data with Deep Transfer Learning},
  author = {George, Daniel and Shen, Hongyu and Huerta, E. A.},
  journal = {Phys. Rev. D},
  volume = {97},
  issue = {10},
  pages = {101501},
  numpages = {7},
  year = {2018},
  month = {May},
  publisher = {American Physical Society},
  doi = {10.1103/PhysRevD.97.101501},
  url = {https://link.aps.org/doi/10.1103/PhysRevD.97.101501}
}


@INPROCEEDINGS{cuoco2018wavelet,  
  author={E. {Cuoco} and M. {Razzano} and A. {Utina}},  
  booktitle={2018 26th European Signal Processing Conference (EUSIPCO)},   
  title={Wavelet-Based Classification of Transient Signals for Gravitational Wave Detectors},   
  year={2018},  
  volume={},  
  number={},  
  pages={2648-2652},
  organization={IEEE},
  URL={https://ieeexplore.ieee.org/abstract/document/8553393},
}

@article{zevin2017gravity,
	doi = {10.1088/1361-6382/aa5cea},
	url = {https://doi.org/10.1088%2F1361-6382%2Faa5cea},
	year = 2017,
	month = {feb},
	publisher = {{IOP} Publishing},
	volume = {34},
	number = {6},
	pages = {064003},
	author = {M Zevin and S Coughlin and S Bahaadini and E Besler and N Rohani and S Allen and M Cabero and K Crowston and A K Katsaggelos and S L Larson and T K Lee and C Lintott and T B Littenberg and A Lundgren and C {\O}sterlund and J R Smith and L Trouille and V Kalogera},
	title = {Gravity Spy: integrating advanced {LIGO} detector characterization, machine learning, and citizen science},
	journal = {Classical and Quantum Gravity},
	abstract = {With the first direct detection of gravitational waves, the advanced laser interferometer gravitational-wave observatory (LIGO) has initiated a new field of astronomy by providing an alternative means of sensing the universe. The extreme sensitivity required to make such detections is achieved through exquisite isolation of all sensitive components of LIGO from non-gravitational-wave disturbances. Nonetheless, LIGO is still susceptible to a variety of instrumental and environmental sources of noise that contaminate the data. Of particular concern are noise features known as glitches, which are transient and non-Gaussian in their nature, and occur at a high enough rate so that accidental coincidence between the two LIGO detectors is non-negligible. Glitches come in a wide range of time-frequency-amplitude morphologies, with new morphologies appearing as the detector evolves. Since they can obscure or mimic true gravitational-wave signals, a robust characterization of glitches is paramount in the effort to achieve the gravitational-wave detection rates that are predicted by the design sensitivity of LIGO. This proves a daunting task for members of the LIGO Scientific Collaboration alone due to the sheer amount of data. In this paper we describe an innovative project that combines crowdsourcing with machine learning to aid in the challenging task of categorizing all of the glitches recorded by the LIGO detectors. Through the Zooniverse platform, we engage and recruit volunteers from the public to categorize images of time-frequency representations of glitches into pre-identified morphological classes and to discover new classes that appear as the detectors evolve. In addition, machine learning algorithms are used to categorize images after being trained on human-classified examples of the morphological classes. Leveraging the strengths of both classification methods, we create a combined method with the aim of improving the efficiency and accuracy of each individual classifier. The resulting classification and characterization should help LIGO scientists to identify causes of glitches and subsequently eliminate them from the data or the detector entirely, thereby improving the rate and accuracy of gravitational-wave observations. We demonstrate these methods using a small subset of data from LIGO’s first observing run.}
}

@article{coughlin2019classifying,
  title = {Classifying the unknown: Discovering novel gravitational-wave detector glitches using similarity learning},
  author = {Coughlin, S. and Bahaadini, S. and Rohani, N. and Zevin, M. and Patane, O. and Harandi, M. and Jackson, C. and Noroozi, V. and Allen, S. and Areeda, J. and Coughlin, M. and Ruiz, P. and Berry, C. P. L. and Crowston, K. and Katsaggelos, A. K. and Lundgren, A. and \O{}sterlund, C. and Smith, J. R. and Trouille, L. and Kalogera, V.},
  journal = {Phys. Rev. D},
  volume = {99},
  issue = {8},
  pages = {082002},
  numpages = {8},
  year = {2019},
  month = {Apr},
  publisher = {American Physical Society},
  doi = {10.1103/PhysRevD.99.082002},
  url = {https://link.aps.org/doi/10.1103/PhysRevD.99.082002}
}


@article{cavaglia2018finding,
  title={Finding the origin of noise transients in LIGO data with machine learning},
  author={Cavaglia, Marco and Staats, Kai and Gill, Teerth},
  journal={arXiv preprint arXiv:1812.05225},
  year={2018},
  url={https://arxiv.org/abs/1812.05225},
}


@article{mukund2017transient,
  title = {Transient classification in LIGO data using difference boosting neural network},
  author = {Mukund, N. and Abraham, S. and Kandhasamy, S. and Mitra, S. and Philip, N. S.},
  journal = {Phys. Rev. D},
  volume = {95},
  issue = {10},
  pages = {104059},
  numpages = {9},
  year = {2017},
  month = {May},
  publisher = {American Physical Society},
  doi = {10.1103/PhysRevD.95.104059},
  url = {https://link.aps.org/doi/10.1103/PhysRevD.95.104059}
}


@article{llorens2019classification,
	doi = {10.1088/1361-6382/ab0657},
	url = {https://doi.org/10.1088%2F1361-6382%2Fab0657},
	year = 2019,
	month = {mar},
	publisher = {{IOP} Publishing},
	volume = {36},
	number = {7},
	pages = {075005},
	author = {Miquel Llorens-Monteagudo and Alejandro Torres-Forn{\'{e}} and Jos{\'{e}} A Font and Antonio Marquina},
	title = {Classification of gravitational-wave glitches via dictionary learning},
	journal = {Classical and Quantum Gravity},
	abstract = {We present a new method for the classification of transient noise signals (or glitches) in advanced gravitational-wave interferometers. The method uses learned dictionaries (a supervised machine learning algorithm) for signal denoising, and untrained dictionaries for the final sparse reconstruction and classification. We use a data set of 3000 simulated glitches of three different waveform morphologies, comprising 1000 glitches per morphology. These data are embedded in non-white Gaussian noise to simulate the background noise of advanced LIGO in its broadband configuration. Our classification method yields a 96% accuracy for a large range of initial parameters, showing that learned dictionaries are an interesting approach for glitch classification. This work constitutes a preliminary step before assessing the performance of dictionary-learning methods with actual detector glitches.}
}


@article{george2017deep,
  title={Deep Transfer Learning: A new deep learning glitch classification method for advanced LIGO},
  author={George, Daniel and Shen, Hongyu and Huerta, EA},
  journal={arXiv preprint arXiv:1706.07446},
  year={2017},
  url={https://arxiv.org/abs/1706.07446},
}

@article{astone2018new,
  title = {New method to observe gravitational waves emitted by core collapse supernovae},
  author = {Astone, P. and Cerd\'a-Dur\'an, P. and Di Palma, I. and Drago, M. and Muciaccia, F. and Palomba, C. and Ricci, F.},
  journal = {Phys. Rev. D},
  volume = {98},
  issue = {12},
  pages = {122002},
  numpages = {11},
  year = {2018},
  month = {Dec},
  publisher = {American Physical Society},
  doi = {10.1103/PhysRevD.98.122002},
  url = {https://link.aps.org/doi/10.1103/PhysRevD.98.122002}
}

@article{colgan2020efficient,
  title = {Efficient gravitational-wave glitch identification from environmental data through machine learning},
  author = {Colgan, Robert E. and Corley, K. Rainer and Lau, Yenson and Bartos, Imre and Wright, John N. and M\'arka, Zsuzsa and M\'arka, Szabolcs},
  journal = {Phys. Rev. D},
  volume = {101},
  issue = {10},
  pages = {102003},
  numpages = {12},
  year = {2020},
  month = {May},
  publisher = {American Physical Society},
  doi = {10.1103/PhysRevD.101.102003},
  url = {https://link.aps.org/doi/10.1103/PhysRevD.101.102003}
}

@INPROCEEDINGS{7952693,  
  author={S. {Bahaadini} and N. {Rohani} and S. {Coughlin} and M. {Zevin} and V. {Kalogera} and A. K. {Katsaggelos}},  booktitle={2017 IEEE International Conference on Acoustics, Speech and Signal Processing (ICASSP)},   title={Deep multi-view models for glitch classification},   year={2017},  volume={},  number={},  pages={2931-2935},}